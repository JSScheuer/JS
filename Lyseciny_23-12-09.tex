\documentclass{beamer}
\usetheme{Boadilla}      % or try Darmstadt, Madrid, Warsaw, ...
  \usecolortheme{beaver} % or try albatross, beaver, crane, ...
  \usefonttheme{default}  % or try serif, structurebold, ...
  \setbeamertemplate{navigation symbols}{}
  \setbeamertemplate{caption}[numbered]
\usepackage{color,soul}
\definecolor{red}{rgb}{1,0,0}
\setulcolor{red}
\usepackage{wrapfig}
\usepackage{wasysym}
\usepackage{tikz}
\usepackage[font=scriptsize,labelfont=bf]{caption}


\usepackage[T1]{fontenc}
\usepackage[polish]{babel}
\usepackage[utf8]{inputenc}


\usepackage{amssymb}
\usepackage{amsmath}
\usepackage{amsthm}
\usepackage{esint}
\usepackage{enumerate}
\usepackage{multimedia}

\usepackage{autonum}



%Symbols
\newcommand{\ti}{\tilde}
\newcommand{\wt}{\widetilde}
\newcommand{\wh}{\widehat}
\newcommand{\bs}{\backslash}
\newcommand{\cn}{\colon}
\newcommand{\sub}{\subset}
\newcommand{\ov}{\overline}
\newcommand{\mr}{\mathring}


\newcommand{\bbN}{\mathbb{N}}
\newcommand{\bbZ}{\mathbb{Z}}
\newcommand{\bbQ}{\mathbb{Q}}
\newcommand{\bbR}{\mathbb{R}}
\newcommand{\bbC}{\mathbb{C}}
\newcommand{\bbS}{\mathbb{S}}
\newcommand{\bbH}{\mathbb{H}}
\newcommand{\bbK}{\mathbb{K}}
\newcommand{\bbD}{\mathbb{D}}
\newcommand{\bbB}{\mathbb{B}}
\newcommand{\bbE}{\mathbb{E}}
\newcommand{\bbM}{\mathbb{M}}


\newcommand{\8}{\infty}

%Greek letters
\newcommand{\al}{\alpha}
\newcommand{\be}{\beta}
\newcommand{\ga}{\gamma}
\newcommand{\de}{\delta}
\newcommand{\ep}{\epsilon}
\newcommand{\ka}{\kappa}
\newcommand{\la}{\lambda}
\newcommand{\om}{\omega}
\newcommand{\si}{\sigma}
\newcommand{\Si}{\Sigma}
\newcommand{\ph}{\phi}
\newcommand{\vp}{\varphi}
\newcommand{\ze}{\zeta}
\newcommand{\vt}{\vartheta}
\newcommand{\Om}{\Omega}
\newcommand{\De}{\Delta}
\newcommand{\Ga}{\Gamma}
\newcommand{\Th}{\Theta}
\newcommand{\La}{\Lambda}
\newcommand{\Ph}{\Phi}
\newcommand{\Ps}{\Psi}

%Mathcal Letters
\newcommand{\cL}{\mathcal{L}}
\newcommand{\cT}{\mathcal{T}}
\newcommand{\cA}{\mathcal{A}}
\newcommand{\cW}{\mathcal{W}}
\newcommand{\cH}{\mathcal{H}}
\newcommand{\cS}{\mathcal{S}}
\newcommand{\cD}{\mathcal{D}}
\newcommand{\cB}{\mathcal{B}}
\newcommand{\cF}{\mathcal{F}}
\newcommand{\cN}{\mathcal{N}}
\newcommand{\cU}{\mathcal{U}}
\newcommand{\cV}{\mathcal{V}}
\newcommand{\cO}{\mathcal{O}}
\newcommand{\cI}{\mathcal{I}}
\newcommand{\cK}{\mathcal{K}}
\newcommand{\cR}{\mathcal{R}}
\newcommand{\cP}{\mathcal{P}}






%Mathematical operators
\newcommand{\INT}{\int_{\O}}
\newcommand{\DINT}{\int_{\d\O}}
\newcommand{\Int}{\int_{-\infty}^{\infty}}
\newcommand{\del}{\partial}
\newcommand{\n}{\nabla}
\newcommand{\II}[2]{\mrm{II}\br{#1,#2}}
\newcommand{\fa}{\forall}
\newcommand{\rt}{\sqrt}




\newcommand{\ip}[2]{\left\langle #1,#2 \right\rangle}
\newcommand{\fr}[2]{\frac{#1}{#2}}
\newcommand{\tfr}[2]{\tfrac{#1}{#2}}
\newcommand{\x}{\times}

\DeclareMathOperator{\dive}{div}
\DeclareMathOperator{\id}{id}
\DeclareMathOperator{\pr}{pr}
\DeclareMathOperator{\Diff}{Diff}
\DeclareMathOperator{\supp}{supp}
\DeclareMathOperator{\graph}{graph}
\DeclareMathOperator{\osc}{osc}
\DeclareMathOperator{\const}{const}
\DeclareMathOperator{\dist}{dist}
\DeclareMathOperator{\loc}{loc}
\DeclareMathOperator{\tr}{tr}
\DeclareMathOperator{\Rm}{Rm}
\DeclareMathOperator{\Rc}{Rc}
\DeclareMathOperator{\Sc}{R}
\DeclareMathOperator{\grad}{grad}
\DeclareMathOperator{\ad}{ad}
\DeclareMathOperator{\argmax}{argmax}
\DeclareMathOperator{\vol}{vol}
\DeclareMathOperator{\Area}{Area}
\DeclareMathOperator{\sgn}{sgn}
\DeclareMathOperator{\Rad}{Rad}
\DeclareMathOperator{\nul}{null}
\DeclareMathOperator{\ind}{ind}
\DeclareMathOperator{\diam}{diam}









%Environments


%\newcommand{\Theo}[3]{\begin{#1}\label{#2} #3 \end{#1}}
\newcommand{\pf}[1]{\begin{proof}#1 \end{proof}}
\newcommand{\eq}[1]{\begin{equation}\begin{alignedat}{2} #1 \end{alignedat}\end{equation}}
\newcommand{\IntEq}[4]{#1&#2#3	 &\quad &\text{in}~#4,}
\newcommand{\BEq}[4]{#1&#2#3	 &\quad &\text{on}~#4}
\newcommand{\br}[1]{\left(#1\right)}
\newcommand{\abs}[1]{\lvert #1\rvert}
\newcommand{\enum}[1]{\begin{enumerate}[(i)] #1 \end{enumerate}}
\newcommand{\enu}[1]{\begin{enumerate}[(a)] #1 \end{enumerate}}
\newcommand{\Matrix}[1]{\begin{pmatrix} #1 \end{pmatrix}}



%Logical symbols
\newcommand{\Ra}{\Rightarrow}
\newcommand{\ra}{\rightarrow}
\newcommand{\hra}{\hookrightarrow}
\newcommand{\Iff}{\Leftrightarrow}
\newcommand{\mt}{\mapsto}

%Fonts
\newcommand{\mc}{\mathcal}
\newcommand{\tit}{\textit}
\newcommand{\mrm}{\mathrm}

%Spacing
\newcommand{\hp}{\hphantom}
\newcommand{\q}{\quad}




%\beamerdefaultoverlayspecification{<+->}

 
\title[Stability for anisotropic energies]{Stability for anisotropic curvature functionals}
\author[J. Scheuer]{Julian Scheuer (Goethe University Frankfurt) \\ joint work with Xuwen Zhang (Xiamen/Frankfurt)
\vspace{-0.5cm}
}
\date[Horní Lyse$\check{c}$iny, 09/12/23]{{\bf{Horní Lyse$\check{c}$iny ``Integral''}}
%\begin{figure}\includegraphics[width=0.3\textwidth]{horni.png}\vspace{-0.3cm}
%\captionsetup{labelformat=empty}
%	\caption{Scan me for the slides !}
%\end{figure}
}


 
\begin{document}

\maketitle


%%%%%%%%%%%%%%%%%%%%%%%%%%%%%%%%%%%%

\begin{frame} \setbeamercovered{invisible}
\frametitle{Closed Soap Bubbles}

\begin{itemize}
		\item[] {\textbf{Isoperimetric problem}}: Determine properties of area minimising surface, given volume constraint. 
		\item[] Round spheres in $\bbR^{n+1}$ are unique closed minimisers of
		\eq{\cR(\Om) = \frac{\Area^{\fr{n+1}{n}}(\del\Om)}{\mrm{Volume}(\Om)}}
 	\item[] Standard variational methods:
	\begin{itemize}
	\item Minimisers of $\cR$ have {\bf{constant mean curvature}} 
	\eq{H=\tr(A) = \sum_{i=1}^{n}\ka_{i}}
	$(\ka_{i})$ are eigenvalues of the Weingarten map $A$, principal curvatures.
	\end{itemize}
\end{itemize}	
\end{frame}

\begin{frame} \setbeamercovered{invisible}
\frametitle{Alexandrov's theorem}

\begin{center}
{\textbf{Is a closed embedded constant mean curvature (CMC) hypersurface of $\bbR^{n+1}$ necessarily a sphere? }}
\end{center}


\begin{itemize}
\item[] Answer: {\textbf{YES!} (Alexandrov\footnote{\emph{A characteristic property of spheres}, Ann. Mat.
  Pura Appl. \textbf{58} (1962), no.~4, 303--315.})}
	\begin{itemize}
		\item Proof: Reflection across moving planes and the maximum principle.
		\item We are going to see another elegant proof today. 
	\end{itemize}
\item[] Relaxed CMC condition: Suppose for some $\de>0$, on a hypersurface $M$
\eq{n-\de\leq H\leq n+\de.}
\item[] Can we conclude
\eq{\dist(M,S)\leq C\ep}
for the unit sphere $S$, a constant $C$ and where 
\eq{\lim_{\de\ra 0}\ep(\de)= 0~?}
\end{itemize}
\end{frame}

\begin{frame} \setbeamercovered{invisible}
\frametitle{The question of stability}

\begin{theorem}[Giulio Ciraolo and Luigi Vezzoni\footnotetext{\emph{A sharp quantitative version of
  {A}lexandrov's theorem via the method of moving planes}, J. Eur. Math. Soc.
  \textbf{20} (2018), no.~2, 261--299.}]
Let $\Om$ be a smooth domain with connected boundary, then {\bf{$\del\Om$ lies within an annulus of thickness $C\osc(H)$}}. $C$ depends on $\abs{\del\Om}$ and a lower bound for interior and exterior balls.
\end{theorem}

 Generalization to spaceforms and other curvature functions
\eq{F=F(\ka_{i})}
was given by Ciraolo/Roncoroni/Vezzoni.\footnote{\emph{Quantitative
  stability for hypersurfaces with almost constant curvature in space forms},
  Ann. Mat. Pura Appl. \textbf{200} (2021), no.~5, 2043--2083.}


\end{frame}




\begin{frame} \setbeamercovered{invisible}
\frametitle{The question of stability}
%\begin{center}
%{\textbf{Is a closed embedded almost constant mean curvature hypersurface of $\bbR^{n+1}$ necessarily close to a sphere? }}
%\end{center}

\begin{theorem}[Rolando Magnanini and Giorgio Poggesi\footnotetext{\emph{On the stability for
  {A}lexandrov's soap bubble theorem}, J. Anal. Math. \textbf{139} (2019),
  no.~1, 179--205.}]
Let $\Om$ be a smooth domain with connected boundary, then $\del\Om$ lies within an annulus of thickness at most $C\|H-H_{0}\|_{L^{1}(\del\Om)}^{\tau_{n}}$, where
\eq{H_{0} = \fr{n}{n+1}\fr{\abs{\del\Om}}{\abs{\Om}},}
 $\tau_{n}$ is a dimensional constant and $C$ depends on few geometric quantities, such as interior and exterior ball conditions.

\end{theorem}
\end{frame}


\begin{frame} \setbeamercovered{invisible}
\frametitle{Ambient anisotropic geometry}

\begin{itemize}
\item $\cW$ (aka Wulff shape) smooth boundary of convex body $\cW_{0}$ containing the origin. 
	\begin{itemize}
		\item Minkowski norm
		\eq{F^{0}(x) = \inf\{s>0\cn x\in s\cW_{0}\}}
		\item Then $\cW = \{F^{0}=1\}.$
	\end{itemize}
\item $q(x) = \fr{1}{2}(F^{0}(x))^{2}$ is smooth, convex and hence 
\eq{\bar g(x):=D^{2}q(x)}
is a metric on $\bbR^{n+1}\bs\{0\}$.
\item Let $F(z)=\sup_{x\in \cW}\ip{x}{z}$ be the support function of $\cW$,
	\begin{itemize}
		\item $\Phi = (D^{\sharp}F)_{|\bbS^{n}}\cn \bbS^{n}\ra \cW$ is an embedding of the Wulff shape.
	\end{itemize}
\end{itemize}
\end{frame}

\begin{frame} \setbeamercovered{invisible}
\frametitle{Induced anisotropic geometry}

\begin{itemize}
\item Let $x\cn M^{n}\ra \bbR^{n+1}$ embedding with $M$ closed manifold, $\ti\nu\cn M\ra \bbS^{n}$ its normal vector field aka Gauss map.
	\begin{itemize}
		\item {\it Anisotropic normal} $\nu = \Phi\circ \ti \nu$ $=$ ``position $\nu(x)$ on the Wulff shape, at which the normal of the Wulff shape equals $\ti\nu(x)$''
	\end{itemize}  
\item Tangent space coincidence: $T_{x}M = T_{\ti\nu(x)}\bbS^{n} = T_{\nu(x)}\cW$.
	\begin{itemize}
		\item Homogeneity of $F^{0}$ implies for all tangent vectors $V\in T_{x}M$,
		\eq{\bar g_{\nu}(\nu,\nu) = 1,\q g_{\nu}(\nu,V) = 0}
		\item {\it Induced anisotropic metric} and {\it second fundamental form}: 
		\eq{g_{ij}(x) = \bar g_{\nu(x)}(\del_{i}x,\del_{j}x),\q h_{ij} = \bar g_{\nu(x)}(\del_{i}\nu,\del_{j}x).}
		\item {\it Anisotropic mean curvature $H = g^{ij}h_{ij} \equiv \tr(A)$.}	
	\end{itemize}
\item {\it Anisotropic volume element} $d\mu = F(\ti \nu)d\vol_{n}$ ($\neq$ volume element induced from $g$).
\end{itemize}
\end{frame}

\begin{frame} \setbeamercovered{invisible}
\frametitle{Level sets}

\begin{itemize}
\item For a function $f$ on $\bbR^{n+1}$, define the {\it $F$-gradient} by
\eq{\n_{F}^{\sharp}f = F(D^{\sharp}f)D^{\sharp}F(D^{\sharp}f).}
	\begin{itemize}
		\item On a regular level set $M$ of $f$ this coincides with the definition
		\eq{\bar g_{\nu(x)}(\n^{\sharp}_{F}f,V) = Df(x)V\q\fa V\in T_{x}M. }
	\end{itemize}
\item Define the {\it $F$-Hessian endomorphism} by
\eq{\n^{2}_{F}f  = D^{2}(\tfr 12 F^{2})(D^{\sharp}f)\circ D^{2}f}
	\begin{itemize}
		\item On a regular level set $M$ of $f$ this coincides with	
			\eq{\n^{2}_{F}f= \tr(\bar g_{\nu(x)}^{-1}\circ D^{2}f)}
		\item Operator degenerates where $Df=0$.
	\end{itemize}
\end{itemize}
\end{frame}



\begin{frame} \setbeamercovered{invisible}
\frametitle{Anisotropic level-set stability}

\begin{theorem}[with Xuwen Zhang\footnotetext{\emph{Stability of the Wulff shape with respect to anisotropic curvature functionals, (2021), {\href{https://arxiv.org/abs/2308.15999}{arxiv:2308.15999}}.}}]
Let $n\geq 2$, $M\sub \bbR^{n+1}$ closed hypersurface, $F$ an elliptic integrand, $\mu(M)=1$. Let $\cU$ be one-sided neighbourhood of $M$, given by level sets of $f\in C^{2}(\bar\cU)$, 
\eq{\bar \cU = \bigcup_{0\leq t\leq \max\abs{f}}M_{t},\q M_{t} = \{\abs{f}=t\},} 
 with $f_{|M}=0$ and $df_{|\bar\cU}>0$.  Let $p>n$ and $\max_{0\leq t\leq \max\abs{f}}\|A\|_{p,M_{t}}\leq C_{0}.$
%Then there exists a constant $C = C(n,p,C_{0})$,
Then
%\eq{\br{\int_{\cU}\abs{\mr{\bar\n}^{2} f}^{p}}^{\fr{1}{p+1}}<\fr{1}{C}\min(\max\abs{f},\min\abs{\bar\n f})^{\fr{p}{p+1}}}
%implies
\eq{\dist(M,\cW)\leq \fr{C(n,p,F,C_{0})}{\min(\max\abs{f},\min\abs{df})^{\fr{p}{p+1}}} \br{\int_{\cU}\abs{\mr{\n}^{2}_{F} f}^{p}}^{\fr{1}{p+1}}  } 
for the Wulff shape corresponding to $F$, provided the RHS is small.
\end{theorem} 


\end{frame}




\begin{frame} \setbeamercovered{invisible}
\frametitle{Some words about the proof}

\begin{itemize}
%\item Key: The curvature of every regular level set $M$ of $f$ can be expressed in terms of $\n^{2}f$:
\item[] If $h$ is the anisotropic $2^{nd}$ fundamental form of any regular level set of $f$, then
\eq{\n^{2}f_{| M}=F(D^{\sharp} f)h.}
\item[] Hence
\eq{F^{2}(D^{\sharp}f)\abs{\mr{A}}^{2}\leq \abs{\mr\n^{2}_{F}f}^{2},  }
\item[] where $\mr{A}$ is the tracefree part of the anisotropic second fundamental form, 
\eq{\abs{\mr{A}}^{2}=c_{n}\sum_{i<j}(\ka_{i}-\ka_{j})^{2}.} 
\end{itemize}
\end{frame}


\begin{frame} \setbeamercovered{invisible}
\frametitle{Some words about the proof}
\begin{itemize}
\item[] Hence $\mr{A}$ can be controlled by $\mr\n^{2}_{F} f$.
\item[] Classical result from (isotropic) hypersurface theory, due to Darboux:
\eq{\mr{A}=0\q\Ra\q M = \mbox{Sphere}.}
	\begin{itemize}
		\item A similar result holds in the anisotropic world\footnotemark \footnotetext{Yijun He and Haizhong Li: \emph{Integral formula of Minkowski type and new characterization of the Wulff shape}, Acta Math. Sin. \textbf{24} (2008), no. 4, 697--704.} 
	\end{itemize}
\item[] Stability versions in the isotropic case are plentyfull and due to De Lellis/M\"uller, Topping, Grosjean.
	\begin{itemize}
		\item In the anisotropic world there is a recent one...
	\end{itemize}
\end{itemize}
\end{frame}

\begin{frame} \setbeamercovered{invisible}
\frametitle{Some words about the proof}

\begin{theorem}[Antonio De Rosa, Stefano Gioffr\'e\footnotetext{\emph{Absence of bubbling phenomena for
  non-convex anisotropic nearly umbilical and quasi-{E}instein hypersurfaces},
  J. Reine Angew. Math. \textbf{780} (2021), 1--40.}]
Let $M\sub \bbR^{n+1}$ be a closed hypersurface, $\cW$ a Wulff shape, $p>n$ and 
\eq{|M| = |\cW|,\q \|A\|_{L^{p}(M)}\leq c_{0}.}
Then there exist $C = C(n,p,F,c_{0})>0$, such that: if
\eq{\|\mr{A}\|_{L^{p}(M)}\leq \tfr 1C,}
then there exists $c\in \bbR^{n+1}$ and a parametrization $\psi\cn \cW \ra M,$
such that
 \eq{\|\psi-\id-c\|_{W^{2,p}(\cW)}\leq C\|\mr{A}\|_{L^{p}(M)}.}
\end{theorem}

 %The Co-area formula relates (locally around $M$) $\|\mr A\|_{L^{2}(M)}$ with $\|\mr{\n }^{2}f\|_{L^{2}(\cU)}$.
\end{frame}




\begin{frame} \setbeamercovered{invisible}
\frametitle{Application I: Anisotropic Heintze-Karcher}
\begin{itemize}
	\item {\textbf{Stability of the domain in the Heintze-Karcher inequality}}. In the Euclidean space $\bbR^{n+1}$, for every domain $\Om$ with mean-convex $\del\Om$:
			\eq{\int_{\del\Om}\fr{n}{H}\geq (n+1)\vol(\Om)}
	with equality precisely when $\Om$ is a ball.
	\begin{itemize}
		\item The same holds in the anisotropic setting, if we integrate w.r.t. the anisotropic area measure.
	\end{itemize}
	
	

\end{itemize}
\end{frame}






\begin{frame} \setbeamercovered{invisible}
\frametitle{Application I: Anisotropic Heintze-Karcher}

\begin{theorem}[with Xuwen Zhang, Stability in the anisotropic Heintze-Karcher]
Let $n\geq 2$, $\al>0$ and $F$ be an elliptic integrand on $\mathbb{R}^{n+1}$.
Let $\Om\sub \bbR^{n+1}$ be a bounded domain with connected $F$-mean convex $C^{2,\al}$-boundary that satisfies a uniform interior Wulff sphere condition with radius $r$. 
Then there exists a positive constant $C$ depending only on $n,\al,F,r,\mu(\del\Om)$ and $\abs{\del\Om}_{2,\al}$, such that
\eq{
\dist(\del\Om,{\cW})
    \leq C
    \left(\int_{\del\Om}\frac{1}{H}\,d\mu-\frac{n+1}{n}\vert\Om\vert\right)^{\frac{1}{n+2}}
}
for some Wulff sphere ${\cW}\subset\mathbb{R}^{n+1}$, provided the RHS is sufficiently small.
%and some positive constant $C$ that depends on $n,F,r_i,\vert\Om\vert,\mu(M)_2,\mu(M),\vert D^2f\vert_{0,\overline{\mathcal{U}}_f}$.
\end{theorem}
\end{frame}




\begin{frame} \setbeamercovered{invisible}
\frametitle{Proof of Heintze-Karcher stability}

\begin{itemize}
\item The key for stability is the following estimate
	\eq{\int_{\Om}\abs{\mr\n^{2}_{F}f}^{2}\,dx\leq \br{\fr{n}{n+1}}^{2}\int_{\del\Om}\fr{1}{H}\,d\mu -\fr{n+1}{n}\abs{\Om},}
	where
	\eq{\dive(D^{\sharp}(\tfr 12 F^{2})(D^{\sharp}f)) =: \De_{F}f&=1\q &&\mbox{in}~\Om\\
	f&=0\q &&\mbox{on}~\del\Om.}
	\begin{itemize}
		\item Follows from divergence theorem type argument and H\"older's inequality.
		\item $f$ shall serve as the foliation function in a neighbourhood of $\del\Om$. 
		\item For this we need a lower gradient bound of $f$ on $\del\Om$.
	\end{itemize}
\end{itemize}
\end{frame}



\begin{frame} \setbeamercovered{invisible}
\frametitle{Proof of Heintze-Karcher stability}


\begin{lemma}[Gradient bound on $\del\Om$]\label{Lem-IWC-bound}
Let $\Om\subset\mathbb{R}^{n+1}$ be a bounded domain with  $C^{2}$-boundary that satisfies the uniform interior Wulff sphere condition with radius $r$ and let $f\in C^{1,\beta}(\overline\Om)\cap W^{2,2}(\Om)$ for some $\beta\in(0,1)$ be a solution of
 \eq{\dive(D^{\sharp}(\tfr 12 F^{2})(D^{\sharp }f)) &=1\q &&\mbox{in}~\Om\\
	f&=0\q &&\mbox{on}~\del\Om.}
then
\eq{
\vert D^\sharp f\vert
    \geq C(n,F)r\quad\text{on }\del\Om.
}
\end{lemma}

\begin{itemize}
\item From here, higher regularity in a controlled neighbourhood of $\del\Om$ follows from Schauder theory.
\item The level set stability theorem completes the proof of the anisotropic Heintze-Karcher stability.
\end{itemize}

\end{frame}


\begin{frame} \setbeamercovered{invisible}
\frametitle{Application II: Stability in the anisotropic soap bubble theorem}

\begin{theorem}[with Xuwen Zhang, Stability in the anisotropic soap bubble theorem]
Let $n\geq 2$, $\al>0$ and $F$ be an elliptic integrand on $\mathbb{R}^{n+1}$.
Let $\Om\sub \bbR^{n+1}$ be a bounded domain with connected boundary $\del\Om\in C^{2,\al}$ that satisfies a uniform interior Wulff sphere condition with radius $r$. 
Then there exists a positive constant $C$ depending only on $n,\al,F,r,\mu(\del\Om)$, such that 
\eq{
\dist(\del\Om,{\cW})
    \leq C
    \left\|H-\tfr{n}{n+1}\tfr{\mu(\del\Om)}{\abs{\Om}}\right\|_{1,\del\Om}^{\frac{1}{n+2}}
}
for some Wulff sphere ${\cW}\subset\mathbb{R}^{n+1}$, provided the RHS is sufficiently small.
\end{theorem}
\begin{itemize}
\item Basically the same proof as in Heintze-Karcher, up to few algebraic modifications. We again estimate $\int_{\Om}\abs{\mr\n^{2}_{F}f}^{2}$.
\end{itemize}

\end{frame}

\begin{frame} \setbeamercovered{invisible}
\frametitle{Application III: Serrin' problem}
\begin{itemize}
\item Serrin's overdetermined problem asks which domains $\Om$ allow solutions to
\eq{\De u & = 1\q \mbox{in}~\Om\\
	u&= 0\q\mbox{on}~\del\Om\\
	\abs{\n u}&=c\q\mbox{on}~\del\Om.}
\begin{itemize}
	\item The answer is: Only balls.
\end{itemize}

\item The anisotropic version (yielding equality to the Wulff shape)
 \eq{\dive(D^{\sharp}(\tfr 12 F^{2})(D^{\sharp }f)) &=1\q &&\mbox{in}~\Om\\
	f&=0\q &&\mbox{on}~\del\Om\\
	F(D^{\sharp}f) &= c\q&&\mbox{on}~\del\Om}
is due to Cianchi/Salani.\footnote{{\it Overdetermined anisotropic elliptic problems}, Math. Ann. {\bf 345} (2009), no. 4, 859--881.}

\end{itemize}


\end{frame}


\begin{frame} \setbeamercovered{invisible}
\frametitle{Application III: Stability in the anisotropic Serrin problem}

\begin{theorem}[with Xuwen Zhang, Stability in the anisotropic Serrin problem]
Let $n\geq 2$, $\al>0$ and $F$ be an elliptic integrand on $\mathbb{R}^{n+1}$.
Let $\Om\sub \bbR^{n+1}$ be a bounded domain with connected boundary $\del\Om\in C^{2,\al}$ that satisfies a uniform interior Wulff sphere condition with radius $r$. 
Then there exists a positive constant depending only on $n,\al, F,r,\diam(\Om)$, $\mu(\del\Om)$ and $\abs{\del\Om}_{2,\al}$, such that
\eq{
\dist(\del\Om,{\cW})
    \leq C
    \left\|F(D^\sharp f)-\tfrac{\vert\Om\vert}{\mu(\del\Om)}\right\|_{1,\del\Om}^{\frac{1}{2(n+2)}}
}
for some Wulff sphere ${\cW}\subset\mathbb{R}^{n+1}$, provided the RHS is sufficiently small. Here 
 \eq{\dive(D^{\sharp}(\tfr 12 F^{2})(D^{\sharp }f)) &=1\q &&\mbox{in}~\Om\\
	f&=0\q &&\mbox{on}~\del\Om.}
	\end{theorem}
\end{frame}

\begin{frame} \setbeamercovered{invisible}
\frametitle{Application III: Stability in the anisotropic Serrin problem}

\begin{itemize}
\item The proof works via the use of a so-called $P$-function. In our setting,
\eq{P=\fr{F^{2}(D^{\sharp}f)}{2}-\fr{1}{n+1}f.}
	\begin{itemize}
		\item A computation gives
			\eq{\dive(\n^{\sharp}_{F}P) = \abs{\mr\n^{2}_{F}f}^{2}.}
	\end{itemize}
\item The next crucial ingredient is the Pohozaev-type identity
\eq{\int_{\Om}P = \fr{1}{2(n+1)}\int_{\del\Om}F^{2}(D^{\sharp}f)\ip{x}{\ti \nu}.}
\item Further computations lead to
\eq{\label{eq-integral-identity-OverDetermined}
    \int_\Om (-f)\vert\mathring{\bar\n}_{F}^2f\vert^2\,d x
    =\frac12\int_{M}\br{F^2(D^\sharp f)-\fr{\abs{\Om}^{2}}{\mu(M)^{2}}}\left<\bar\nabla^\sharp_Ff-\bar\nabla^\sharp_F\ell,\tilde\nu\right>\,d\tilde\mu,}
    where $\ell(x) = (F^{0}(x))^{2}/(2(n+1)).$
    \end{itemize}

\end{frame}



























 







\begin{frame}\setbeamercovered{invisible}
\begin{center}\Huge D\^{e}kuji! \end{center}

 \end{frame}















\bibliographystyle{amsplain}
\bibliography{/Users/julianscheuer/Documents/Uni/TexTemplates/Bibliography.bib}


\end{document}








